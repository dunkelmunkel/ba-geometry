\documentclass{../cssheet}

%--------------------------------------------------------------------------------------------------------------
% Basic meta data
%--------------------------------------------------------------------------------------------------------------

\title{Beweisen}
\author{Prof. Dr. Christian Spannagel}
\date{\today}
\setsubject{Aufgabenblatt Geometrie}
\setkeywords{geometrie}
\setpdfmetadata



%--------------------------------------------------------------------------------------------------------------
% document
%--------------------------------------------------------------------------------------------------------------

\begin{document}
\printtitle

\textbf{Aufgabe 1 (Geraden):}  Beweist die folgenden Aussagen aus den Inzidenzaxiomen:
\begin{enumerate}[a)]
\item Es seien $g$ und $h$ zwei Geraden. Wenn $g$ und $h$ nicht identisch sind, haben sie höchstens einen Punkt gemeinsam.
\item Es seien $g$ und $h$ zwei Geraden. Wenn $g$ und $h$ mehr als einen Punkt gemeinsam haben, so sind sie dentisch.
\item Es existieren mindestens drei paarweise verschiedene Geraden.
\end{enumerate}

\textbf{Aufgabe 2 (Winkelzeugs):}  Beweist die folgenden Aussagen:
\begin{enumerate}[a)]
\item Scheitelwinkel sind gleich groß.
\item Wechselwinkel sind gleich groß.
\end{enumerate}


\textbf{Aufgabe 3 (Innenwinkelsumme Dreieck):}  Beweist: Die Innenwinkelsumme eines Dreiecks beträgt $180^\circ$.

\textbf{Aufgabe 4 (Mittelsenkrechte):} Eine Mittelsenkrechte einer Strecke $\overline{AB}$ ist die Gerade, die durch den Mittelpunkt von $\overline{AB}$ geht und senkrecht auf $\overline{AB}$ steht.
\begin{enumerate}[a)]
\item Beweist: Jeder Punkt $P$ auf der Mittelsenkrechten einer Strecke $\overline{AB}$ hat denselben Abstand von $A$ und $B$.
\item Formuliert die Umkehrung des Satzes aus a) und beweist sie.
\end{enumerate}

\textbf{Aufgabe 5 (Basiswinkel):} 
\begin{enumerate}[a)]
\item Beweist: In einem gleichschenkligen Dreieck sind die Basiswinkel gleich groß.
\item Formuliert die Umkehrung des Satzes aus a) und beweist sie.
\end{enumerate}

\textbf{Aufgabe 6 (Seiten und Winkel im Dreieck):} 
In dieser Aufgabe sollt ihr die Beziehungen zwischen Seitenlängen und Innenwinkelgrößen von Dreiecken untersuchen.
\begin{enumerate}[a)]
\item Beweist: Der größeren Seite liegt der größere Winkel gegenüber.
\item Beweist: Dem größeren Winkel liegt die größere Seite gegenüber.
\end{enumerate}

\textbf{Aufgabe 7 (Winkelhalbierende):} Eine Winkelhalbierende eines Winkels ist die Halbgerade, die durch den Scheitelpunkt des Winkels läuft und das Winkelfeld in zwei deckungsgleiche Teile teilt.

\begin{enumerate}[a)]
\item Beweist: Wenn ein Punkt $P$ auf der Winkelhalbierenden zweier Geraden $g$ und $h$ liegt, dann sind die Abstände von $g$ zu $P$ und von $h$ zu $P$ gleich.
\item Formuliert die Umkehrung des Satzes aus a) und beweist sie.
\end{enumerate}

\newpage
%\vspace*{10mm}
\printlicense

\printsocials

%\pagestyle{docstyle}
\end{document}
