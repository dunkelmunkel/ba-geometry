\documentclass{../cssheet}

%--------------------------------------------------------------------------------------------------------------
% Basic meta data
%--------------------------------------------------------------------------------------------------------------

\title{Aufgaben zum Aufwärmen}
\author{Prof. Dr. Christian Spannagel}
\date{\today}
\setsubject{Aufgabenblatt Geometrie}
\setkeywords{geometrie}
\setpdfmetadata

%--------------------------------------------------------------------------------------------------------------
% document
%--------------------------------------------------------------------------------------------------------------

\begin{document}

\printtitle

\textbf{Vorbemerkungen:} 
\begin{enumerate}
\item Ein wichtiges didaktisches Grundprinzip ist das Anknüpfen neuer Inhalte an Vorwissen. Ihr bringt jede Menge geometrisches Vorwissen aus der Schule mit, aber vielleicht habt ihr das ein oder andere wieder vergessen. Daher ist es wichtig, das Vorwissen zu \emph{aktivieren}. Das ist das Ziel dieser Aufwärm-Aufgaben.
\item Bearbeitet die Aufgaben in eurer Lerngruppe, aber ohne dabei irgendwelche Hilfsmittel zu verwenden! Sucht keine Lösungen im Internet, sondern diskutiert in der Gruppe eure Ideen und korrigiert euch gegenseitig. Dieser gemeinsame Prozess voller unterschiedlicher Gedanken, in denen auch Fehler gemacht und korrigiert werden, steckt voller wertvoller Informationen. Ihr erinnert euch dann später, warum ihr bestimmte Ideen verworfen habt und warum ihr euch für andere Ansätze entschieden habt. Wenn ihr einfach nur die Lösungen googelt oder ChatGPT verwendet, beraubt ihr euch selbst dieser wertvollen Informationen.
\end{enumerate}


\textbf{Aufgabe 1 (Vierecke):}  
\begin{enumerate}[a)]
\item Beschreibt möglichst genau, wodurch die folgenden Vierecke gekennzeichnet sind: Trapez, Raute, Rechteck, Parallelogramm, Quadrat, Drachenviereck.
\item Welche Eigenschaften haben die jeweiligen Vierecke? Sammelt alle Eigenschaften, die euch einfallen!
\item \glqq{}Ein Quadrat ist ein spezielles Rechteck.\grqq{}. Sammelt weitere Aussagen dieser Art und versucht einmal, den Zusammenhang der verschiedenen Vierecke zu visualisieren.
\end{enumerate}

\textbf{Aufgabe 2 (Dreiecke):} Macht genau das gleiche wie in Aufgabe~1, diesmal nur mit folgenden Dreiecken: stumpfwinkliges Dreieck, gleichseitiges Dreieck, rechtwinkliges Dreieck, gleichwinkliges Dreieck, gleichschenkliges Dreieck, spitzwinkliges Dreieck

\textbf{Aufgabe 3 (Weitere geometrische Objekte):} Versucht einmal, folgende geometrische Objekte möglichst genau zu charakterisieren: Punkt, Strecke, Gerade, Halbgerade, Kreis.

\textbf{Aufgabe 4 (Kaffeesorten):} Was ist ein Cappuccino? Was ein Milchkaffee? Was ein Latte Macchiato? Versucht einmal, die verschiedenen Kaffeegetränke eindeutig zu charakterisieren. Fallen euch weitere Kaffeegetränke ein?


\vspace*{10mm}
\printlicense

\printsocials


%\pagestyle{docstyle}
\end{document}
